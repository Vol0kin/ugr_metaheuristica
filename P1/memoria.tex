\documentclass[11pt,a4paper]{article}
\usepackage[spanish]{babel}					% Utilizar español
\usepackage[utf8]{inputenc}					% Caracteres UTF-8
\usepackage{graphicx}						% Imagenes
\usepackage[hidelinks]{hyperref}			% Poner enlaces sin marcarlos en rojo
\usepackage{fancyhdr}						% Modificar encabezados y pies de pagina
\usepackage{float}							% Insertar figuras
\usepackage[textwidth=390pt]{geometry}		% Anchura de la pagina
\usepackage[nottoc]{tocbibind}				% Referencias (no incluir num pagina indice en Indice)
\usepackage{amsmath}

% Configuracion de encabezados y pies de pagina
\pagestyle{fancy}
\lhead{Vladislav Nikolov Vasilev}
\rhead{Metaheurísticas}
\lfoot{Grado en Ingeniería Informática}
\cfoot{}
\rfoot{\thepage}
\renewcommand{\headrulewidth}{0.4pt}		% Linea cabeza de pagina
\renewcommand{\footrulewidth}{0.4pt}		% Linea pie de pagina
\decimalpoint								% Hacer que los números decimales usen . en vez de ,

\begin{document}
\pagenumbering{gobble}

% Pagina de titulo
\begin{titlepage}

\begin{minipage}{\textwidth}

\centering

\includegraphics[scale=0.3]{img/ugr.png}\\

\textsc{\Large Metaheurísticas\\[0.2cm]}
\textsc{GRADO EN INGENIERÍA INFORMÁTICA}\\[0.3cm]

\noindent\rule[-1ex]{\textwidth}{1pt}\\[1.5ex]
\textsc{{\Huge PRÁCTICA 1\\[0.1cm]}}
\textsc{{\Large \\Problema del Aprendizaje de Pesos en Características (APC)}}
\noindent\rule[-1ex]{\textwidth}{2pt}\\[3.5ex]

\end{minipage}

\vspace{0.5cm}

\begin{minipage}{\textwidth}

\centering

\textbf{Autor}\\ {Vladislav Nikolov Vasilev}\\[1ex]
\textbf{NIE}\\ {X8743846M}\\[1ex]
\textbf{E-Mail}\\ {vladis890@gmail.com}\\[1ex]
\textbf{Grupo de prácticas}\\ {MH3 Jueves 17:30-19:30}\\[1ex]
\textbf{Rama}\\ {Computación y Sistemas Inteligentes}\\[1ex]
\vspace{0.2cm}

\includegraphics[scale=0.3]{img/etsiit.jpeg}

\vspace{0.3cm}
\textsc{Escuela Técnica Superior de Ingenierías Informática y de Telecomunicación}\\
\vspace{1cm}
\textsc{Curso 2018-2019}
\end{minipage}
\end{titlepage}

\pagenumbering{arabic}
\tableofcontents
\thispagestyle{empty}				% No usar estilo en la pagina de indice

\newpage

\setlength{\parskip}{1em}

\section{Descripción del problema}

El problema que se aborda en esta práctica es el Aprendizaje de Pesos en Características (APC). Es un problema típico de
\textit{machine learning} en el cuál se pretende optimizar el rendimiento de un clasificador basado en vecinos más cercanos.
Esto se consigue mediante la ponderación de las características de entrada con un vector de pesos $W$, el cuál utiliza
codificación real (cada $w_i \in W$ es un número real), con el objetivo de modificar sus valores a la hora de calcular la
distancia. Cada vector $W$ se expresa como $W = \lbrace w_1, w_2, \cdots , w_n \rbrace$, siendo $n$ el número de dimensiones
del vector de características, y cumpliéndose además que $\forall w_i \in W, \; w_i \in [0, 1]$.\par

El clasificador considerado para este problema es el 1-NN (genéricamente, un clasificador $k$-NN, con $k$ vecinos, siendo
en este caso $k = 1$), es decir, aquél que clasifica un elemento según su primer vecino más cercano utilizando alguna medida
de distancia (en este caso, utilizando la distancia Euclídea). Cabe destacar que no en todos los casos se usará el clasificador
1-NN ya que se pueden dar casos en los que el vecino más cercano de un elemento sea él mismo. Por ese motivo, en algunas
técnicas/algoritmos se usará un 1-NN con el criterio de \textit{leave-one-out}, es decir, que se busca el vecino más cercano
pero excluyéndose a él mismo.\par

El objetivo propuesto es aprender el vector de pesos $W$
mediante una serie de algoritmos, de tal forma que al optimizar el clasificador se mejore tanto la precisión de éste como su
complejidad, es decir, que se considere un menor número de características. Estos dos parámetros, a los que llamaremos $tasa\_
clas$ y $tasa\_red$, respectivamente, se pueden expresar de la siguiente forma:

\[tasa\_clas = 100 \cdot \frac{nº \; instancias \; bien \; clasificadas \; en \; T}{nº \; instancias \; en \; T}\]
\[tasa\_red = 100 \cdot \frac{nº \; valores \; w_i < 0.2}{nº \; caracteristicas}\]

\noindent siendo $T$ el tamaño del conjunto de datos sobre que el que se evalúa el clasificador.\par

Por tanto, al combinarlos en una única función a la que llamaremos $F(W)$, la cuál será nuestra función objetivo a optimizar,
tenemos que:

\[F(W) = \alpha \cdot tasa\_clas(W) + (1 - \alpha) \cdot tasa\_red(W)\] 

\noindent siendo $\alpha$ la importancia que se le asigna a la tasa de clasificación y a la de reducción, cumpliendo que
$\alpha \in [0, 1]$. En este caso, se utiliza un $\alpha = 0.5$ para dar la misma importancia a ambos, con lo cuál se pretende
que se reduzcan al máximo el número de características conservando una $tasa\_clas$ alta.

\section{Descripción de los algoritmos}

\section{Descripción del método de búsqueda}

\newpage

\begin{thebibliography}{5}

\bibitem{nombre-referencia}
Texto referencia
\\\url{https://url.referencia.com}

\end{thebibliography}

\end{document}

